%% 第一章--chapter1.tex
\chapter{模板使用之前}
\section{概述}
BUCTthesis是北京化工大学本科毕业论文的 \LaTeX\ 写作模板,本文是模板的使用指南,
它是由该模板编译而成的。本指南的目的在于帮助北化本科毕业生掌握此模板的使用方法,
从而写出符合北化\href{https://jiaowuchu.buct.edu.cn/2018/1009/c515a22046/page.htm}%
{《本科生毕业设计(论文)撰写规范》}(下称《规范》)要求的毕业论文。

本模板的内容范围仅限于毕业设计(论文)文本,且不含论文封面。开题报告、任务书、中期检查表、
评阅意见表及其说明、评分手册与优秀毕业设计(论文)简介等非毕业设计(论文)文本部分不在其中。

本写作模板基于\CTeX{}的book 文类,默认编码为 UTF-8。本模板已在Windows\ 10(64bit)操作系统下,
\TeX\ Live 2019,使用 XeLaTeX 编译通过,由于作者精力所限,模板未在Mac OS和Linux系统下测试,
故不能保证其可行性。由于此模板的参考文献使用\BibTeX{},所以完整的编译链为:
XeLaTeX $\rightarrow$ BibTeX{}$\rightarrow$ XeLaTeX{}$\rightarrow$ XeLaTeX{}。
为检查编译环境,请在开始前对模板的源文件执行一次全编译,若无报错、无警告则合适。

\subsection{免责声明}
请您注意,至本文档编译时,北京化工大学教务处仅提供《规范》而未对本模板做任何测试或授权。
模板作者自当尽力,但限于软件等各种因素,由本模板生成的文档可能仍与要求有所出入,
故不保证审查老师对格式不提意见。在开始使用之前,您同意,任何由于使用本模板而引起的论文格式审查问题
均与本模板作者无关。

\subsection{开源说明}
本项目遵循~\href{https://www.latex-project.org/lppl.txt}{\LaTeX\ Project Public License} 1.3c
或更高版本,代码托管于 \href{https://github.com/Miracle0565/BUCTthesis}{GitHub} 上。对于模板的
任何问题或新功能需求请于此处提交Issue。

由于模板建设尚在起步阶段,欢迎任何有兴趣的同学加入模板的开发工作。

\subsection{模板组成}
在表 \ref{tab:mainfile} 中罗列了本模板所包含的文件。
\begin{table}%[H]
	\centering
	\caption{主要文件}
	\label{tab:mainfile}
	\begin{tabular}{ll}
		\toprule
		文件(夹)名                & 简述                             \\
		\midrule
		\textsf{buctthesis.tex}   & 主文件                           \\
		\textsf{buctthesis.sty}   & \LaTeX{}宏包文件                 \\
		\textsf{buctthesis.pdf}   & 使用指南,即您正在阅读的这个文件 \\
		\textsf{gbt7714-2005.bst} & \BibTeX{}用到的参考文献格式模板  \\
		\textsf{thesisbib.bib}    & \BibTeX{}参考文献数据库文件      \\
		\textsf{chapter/}         & 各个“章”的源文件路径             \\
		\textsf{code/}            & 源代码的路径                     \\
		\textsf{figure/}          & 图片的路径                       \\
		\textsf{scanPDF/}         & 存放已扫描的封面等文件的路径     \\
		\bottomrule
	\end{tabular}
\end{table}
以下是各个文件(夹)的详细介绍:
\subsubsection{\textsf{buctthesis.tex}}主文件。
内含少量注释,一般您需要做的有:
添加论文相关的信息、增添正文的章节、增添附录的章节。
具体方式请参考第 \ref{chap:CodeIntro} 章。
编译此文件以形成PDF。
\subsubsection{\textsf{buctthesis.sty}}
\LaTeX{}宏包文件,起到格式控制作用。
该文档已经做了部分注释,若您觉得依此宏包编译出的文档有不合理的地方,
可以在其中对应处做些许修改。
\subsubsection{\textsf{gbt7714-2005.bst}}
参考文献格式控制文件,使得参考文献符合规范。此文件包含其制作者信息。
\subsubsection{\textsf{chapter/}、\textsf{figure/}和\textsf{code/}}
这三个文件夹分别存放对应的文件,在第 \ref{chap:CodeIntro} 章会讲述如何将各个章节、
图片和源代码等插入至文章的相应位置中。

\section{一些编辑器的初始配置}
开始之前,最好能使用最新的发行版\LaTeX{},使用旧版可能会有潜在的问题;
同时查看环境变量(此电脑$\rightarrow$ 属性$\rightarrow$ 高级系统设置$\rightarrow$
环境变量$\rightarrow$ 系统变量\,中),
请确保\texttt{C:\cmd{Windows}\cmd{system32}}\footnote{盘符 C 表示系统所在盘}位于其中。

以下仅介绍 Win10 系统下一些常用的编辑器的基本配置。
\subsection{\href{http://texstudio.sourceforge.net}{\TeX{} Studio}}
编译时选择\XeLaTeX{}:
\emph{选项(O)} $\rightarrow$ \emph{设置 TeX Studio} $\rightarrow$
\emph{构建} $\rightarrow$ \emph{默认编译器} 中选择\texttt{XeLaTeX},
编译时会根据实际情况判断是否执行一个完整的编译链。

\subsection{\href{https://code.visualstudio.com}{Visual Studio Code}}
初次使用做如下配置:
\begin{itemize}
	\item 将\TeX{}Live下的 \texttt{\dots \cmd{textlive}\cmd{2019}
			\cmd{bin}\cmd{win32}} 加入至系统环境变量之中,这是您所安装 \TeX{}Live 的路径;
	\item 在\emph{应用商店}(Extensions)中下载扩展:\textsf{LaTeX Workshop},
			您也可以在\href{https://github.com/James-Yu/LaTeX-Workshop}{这里}%
			查看关于此扩展的详细说明;
	\item 在\emph{设置}(Settings)中的搜索框输入 LaTeX,单击搜索显示的第一项
			\emph{在setting.json中编辑},并添加\textsf{code/setting.json}的代码。
\end{itemize}
进一步的设置可以参考\href{https://github.com/EthanDeng/vscode-latex/}{这份文档}。

编译时,在工具栏中\texttt{COMMANDS}内\texttt{Build LaTeX project}选择所需的编译方式。
已配置默认编译为 XeLaTeX,且可手动选择完整的编译链。

若您偏好于其它的一些编辑器如 \href{http://www.lyx.org/}{LyX},在线编译器如
\href{https://www.overleaf.com/}{Overleaf},或是 \TeX\ Live 自带的 TeXworks,
因作者能力有限,无法一一介绍其配置及操作,还请您自己在网上寻求更多帮助。