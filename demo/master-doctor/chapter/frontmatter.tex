%% 前置部分--frontmatter.tex
%% 诚信声明
\makedeclare%[figure/declare-master-doctor.png]

%% 摘要
\begin{cabstract}
	摘要和关键词一起写在这里。

	摘要介绍所研究的主要内容、方法、结果及创新点。应有中文、外文两种文本,
	如无特殊情况,外文文本摘要一般使用英文撰写。中文、外文摘要各占A4纸页面的一半。
	外文摘要要与中文摘要相呼应,其写作模式同中文基本相同。中文摘要一般为300字左右,
	英文摘要为1500印刷符号左右,含中、英文摘要关键词。

	本项目的创新点有:
	\begin{enumerate}
		\item 开发了第一份适用于北京化工大学本科生毕业论文的\LaTeX{}模板;
		\item 以自身为示例展示此模板的使用方法;
		\item 这是编号列表环境的第三项。
	\end{enumerate}

	这里是一些废话,用于填充文本,在后面的部分也会以类似的方式插入无关的文字。
	这一段话的作用是将中文摘要写到 300 个字。

	(这里总共约 300 字)
\end{cabstract}

\begin{eabstract}
	Here is the Abstract and the Keywords.

	In the abstract, you may introduce the main contents of your research,
	as well as the methods, results and some innovation points. There should
	be one Chinese version and one foreign version, while the foreign one
	should generally be written in English except for some special occasions.
	Each of the context of them takes up half of an A4-page. What's more,
	the foreign abstract should be in correspondence to the Chinese one,
	and the narrative pattern is supposed to be similar. The Chinese abstract
	adds up to normally 300 Chinese characters, while the English abstract
	totaled around 1500 printed characters, including Chinese and English keywords.

	Oh, here're just 900 letters total. So I have to add something nonsense\dots.

	One dollar and eighty-seven cents. That was all. And sixty cents of it was
	in pennies. Pennies saved one and two at a time by bulldozing the grocer
	and the vegetable man and the butcher until one's cheeks burned with the
	silent imputation of parsimony that such close dealing implied. Three times
	Della counted it. One dollar and eighty-seven cents. And the next day would
	be Christmas. There was clearly nothing to do but flop down on the shabby
	little couch and howl. So Della did it.
	\hfill --- \textit{THE GIFT OF THE MAGI by O.Henry}

	Innovations in the research:
	\begin{itemize}
		\item Developing the first \LaTeX{} writting template for BUCT undergraduate thesis;
		\item Using the PDF itself as an example to show how to use the template;
		\item This is the third item of an unnumbered list.
	\end{itemize}

	(Around 1500 letters total)
\end{eabstract}