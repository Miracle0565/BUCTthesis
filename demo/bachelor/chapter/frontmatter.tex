%% 前置部分--frontmatter.tex
%% 诚信声明
\makedeclare%[figure/scan-declare.png]

%% 任务书
\begin{taskbook}
\taskinfo		% or use `\taskinfo*' for less lines.

\taskitem		% 1.设计(论文)的主要任务及目标
大学之道,在明明德,在亲民,在止于至善。知止而后有定;
定而后能静;静而后能安;安而后能虑;虑而后能得。
物有本末,事有终始。知所先后,则近道矣。

古之欲明明德于天下者,先治其国;欲治其国者,先齐其家;
欲齐其家者,先修其身;欲修其身,先正其心;欲正其心者,先诚其意;
欲诚其意者,先致其知;致知在格物。物格而后知至;知至而后意诚;
意诚而后心正;心正而后身修;身修而后家齐;家齐而后国治;
国治而后天下平。

自天子以至于庶人,壹是皆以修身为本。其本乱而未治者否矣。
其所厚者薄,而其所薄者厚,未之有也。此谓知本,此谓知之至也。
	
\taskitem		% 2.设计(论文)的基本要求和内容
古之学者必有师。师者,所以传道受业解惑也。人非生而知之者,孰能无惑?
惑而不从师,其为惑也,终不解矣。生乎吾前,其闻道也固先乎吾,吾从而
师之;生乎吾後,其闻道也亦先乎吾,吾从而师之。吾师道也,夫庸知其年
之先後生於吾乎!是故无贵无贱无长无少,道之所存,师之所存也。
	
\taskitem		% 3.主要参考文献
	\begin{bibenumerate}
		\item 北京化工大学教务处. 本科生毕业设计(论文)撰写规范 [EB/OL]. 2018[2020-04-08]. \url{https://jiaowuchu.buct.edu.cn/2018/1009/c515a22046/page.htm}.
		\item 刘海洋. \LaTeX\ 入门 [M]. 北京 : 电子工业出版社, 2013.
		\item MITTELBACH F, GOOSSENS M, BRAAMS J, et al. The \LaTeX\ Companion[M]. 2nd ed. Reading, Massachusetts : Addison-Wesley, 2004.
		\item 
	\end{bibenumerate}
	
\taskitem		% 4.进度安排
	\begin{table}[H]
		\centering
		\begin{tabularx}{.95\textwidth}{p{1.5em}|X|p{6em}}
			\hline
					& 	设计(论文)各阶段名称		  &		起止日期	\\\hline
				1	& 								&				\\\hline
				2	&								&				\\\hline
				3	&								&				\\
			\hline
		\end{tabularx}
	\end{table}
\end{taskbook}


%% 摘要
\begin{cabstract}
	摘要和关键词一起写在这里。

	摘要介绍所研究的主要内容、方法、结果及创新点。应有中文、外文两种文本,
	如无特殊情况,外文文本摘要一般使用英文撰写。中文、外文摘要各占A4纸页面的一半。
	外文摘要要与中文摘要相呼应,其写作模式同中文基本相同。中文摘要一般为300字左右,
	英文摘要为1500印刷符号左右,含中、英文摘要关键词。

	本项目的创新点有:
	\begin{enumerate}
		\item 开发了第一份适用于北京化工大学本科生毕业论文的\LaTeX{}模板;
		\item 以自身为示例展示此模板的使用方法;
		\item 这是编号列表环境的第三项。
	\end{enumerate}

	这里是一些废话,用于填充文本,在后面的部分也会以类似的方式插入无关的文字。
	这一段话的作用是将中文摘要写到 300 个字。

	(这里总共约 300 字)
\end{cabstract}

\begin{eabstract}
	Here is the Abstract and the Keywords.

	In the abstract, you may introduce the main contents of your research,
	as well as the methods, results and some innovation points. There should
	be one Chinese version and one foreign version, while the foreign one
	should generally be written in English except for some special occasions.
	Each of the context of them takes up half of an A4-page. What's more,
	the foreign abstract should be in correspondence to the Chinese one,
	and the narrative pattern is supposed to be similar. The Chinese abstract
	adds up to normally 300 Chinese characters, while the English abstract
	totaled around 1500 printed characters, including Chinese and English keywords.

	Oh, here're just 900 letters total. So I have to add something nonsense\dots.

	One dollar and eighty-seven cents. That was all. And sixty cents of it was
	in pennies. Pennies saved one and two at a time by bulldozing the grocer
	and the vegetable man and the butcher until one's cheeks burned with the
	silent imputation of parsimony that such close dealing implied. Three times
	Della counted it. One dollar and eighty-seven cents. And the next day would
	be Christmas. There was clearly nothing to do but flop down on the shabby
	little couch and howl. So Della did it.
	\hfill --- \textit{THE GIFT OF THE MAGI by O.Henry}

	Innovations in the research:
	\begin{itemize}
		\item Developing the first \LaTeX{} writting template for BUCT undergraduate thesis;
		\item Using the PDF itself as an example to show how to use the template;
		\item This is the third item of an unnumbered list.
	\end{itemize}

	(Around 1500 letters total)
\end{eabstract}